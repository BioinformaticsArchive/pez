\documentclass[11pt]{article} % Default font size is 12pt, it can be changed hee
\usepackage[utf8]{inputenc}
\usepackage[top=0.5in, bottom=0.5in, left=0.5in, right=0.5in]{geometry} % Required to change the page size to A4
\geometry{a4paper} % Set the page size to be A4 as opposed to the default US Letter
\usepackage{graphicx} % Required for including pictures
\usepackage[citestyle=authoryear,bibstyle=authoryear,sorting=nyt,maxcitenames=3,doi=false,url=false,isbn=false,firstinits=true,uniquename=false, uniquelist=false]{biblatex}
\addbibresource{lib.bbl}
\renewbibmacro*{name:andothers}{% Based on name:andothers from biblatex.def
  \ifboolexpr{
    test {\ifnumequal{\value{listcount}}{\value{liststop}}}
    and
    test \ifmorenames
  }
    {\ifnumgreater{\value{liststop}}{1}
       {\finalandcomma}
       {}%
     \andothersdelim\bibstring[\emph]{andothers}}
    {}}
\renewcommand*{\finalnamedelim}{%
  \ifnumgreater{\value{liststop}}{2}{\finalandcomma}{}%
  \addspace\&\space}
\renewbibmacro{in:}{}
\AtEveryBibitem{%
  \clearfield{day}%
  \clearfield{month}%
  \clearfield{endday}%
  \clearfield{endmonth}%
  \clearfield{DOI}%
  \clearfield{Doi}%
  \clearfield{Eprint}%
}
\DeclareFieldFormat[article]{citetitle}{#1}
\DeclareFieldFormat[article]{title}{#1}
\usepackage{url}

\begin{document}
\section*{Title Page}
pez: Phylogenetics for the Environmental Sciences   %Do you want to capitalize Pez? I would keep it as pez?
\section*{Short Structured Abstract}
\subsection*{Summary}
Pez is an \emph{R} package that calculates >30 community phylogenetic 
metrics, statistically models phylogenetic, trait, and community data,
and simulates community structure. It provides a common programmatic
standard for the management and manipulation of eco-phylogenetic data  %Define eco-phylogenetic or do not use it here at the beginning?
in \emph{R}.
\subsection*{Availability}`
Pez is released under the GPL v3 open-source license, available on the
Internet from CRAN (\url{http://cran.r-project.org}). The package is
under active development, and the authors welcome contributions (see
\url{http://github.com/willpearse/pez}).
\subsection*{Contact}
William D.\ Pearse (\url{wdpearse@umn.edu}) Department of Ecology, Evolution, and Behavior, University of Minnesota, 100 Ecology Building, 1987 Upper Buford Circle, Saint Paul, Minnesota, 55108, USA.
\\Marc Cadotte XXX
\\Caroline Tucker XXX
\\Steve Walker XXX
\\Matthew R.\ Helmus (\url{mrhelmus@gmail.com}) Department of Animal Ecology, Vrije Universiteit, 1081 HV, Amsterdam, The Netherlands
\subsection*{Supplmentary information}
A vignette (available with the package) showing some of the package's
functions, and briefly describing the classification of community
phylogenetic metrics.
\section*{Text}
\subsection*{Introduction}
Community phylogenetics is a subfield of ecology which links ecological phenomena (e.g., competition, ecosystem functioning)
with the evolutionary processes that generate species and their traits
\autocite[see][]{Webb2002,Cavender-Bares2009}. This growing field has
generated a number of statistical tools, mostly metrics of phylogenetic biodiversity, and software code and packages to implement such statistics
\autocite[notably][]{Webb2008,Regetz2009,Kembel2010,Orme2013,Eastman2013}. The code is disparate and handles data differently 
making routine data analyses with different statistcs by the end user challenging. In many cases, the statistics are also not formally implmeneted in a particular software package and are available only as supplementary materials to papers which are not
maintained or have public source code. Without active maintenance, these valuable statistics are effectively lost to
the scientific community.

\emph{Pez} provides an \emph{R} class (\emph{comparative.comm}) that
integrates phylogenetic, community, environmental and trait data in as
a single object. From one of these eco-phylogenetic data objects, one can calculates a large body of phylogenetic biodiversity metrics (many
previously unavailable in \emph{R}). The implemented metrics are thematically organized according to the framework outlined in \textcite{Pearse2014review} alowing the user to better understand the actual aspects of phylogenetic biodiversity they are calculating with any particular metric. As the structure of the package has been written to be flexible and intuative, newly published metrics will continue to be incorporated in \emph{Pez}. 

Further, \emph{Pez} implements and extends the regression framework presented
by \parencite{Cavender-Bares2004} \autocite[see
also][]{Cavender-Bares2006}. Finally, it provides functions that simulate ecological communities to exhibit varying degrees of phylogenetic and trait community structure. Together, the functions we have implemented in \emph{Pez} will facilitate and speed the use of analysis of eco-phylogenetic
datasets, and provide a common \emph{R} framework for community phylogenetics.
\subsection*{Description}
\subsubsection*{Data manipulation and storage}
\emph{Pez} extends the data manipulation and import/export features of
\emph{R}, and contains a unified class to contain all eco-phyogenetic
data. It simplifyies the process of ensuring community, trait,
environmental, and phylogenetic data are compatible by overloading
\emph{R}'s standard row and column operators. This makes it simple to
make consistent changes across datasets; for example, removing a
species from the community data automatically trims it from the
phylogeny, and removes its associated trait data. \emph{Pez} was
designed to be backwards-compatible with \emph{caper}'s
\autocite{Orme2013} \emph{comparative.data} class, seamlessly permitting detailed
comparative analysis of species trait data.
\subsubsection*{Metrics}
Following the classification of \textcite{Pearse2014review},
\emph{pez} simplifies the calculation and comparison of a number of
metrics by grouping them into four categories: \emph{shape},
\emph{evenness}, \emph{dispersion}, and \emph{dissimilarity}. Shape metrics measure the structure of an community phylogeny. Evenness metrics are shape metrics that incorporate a weighting with species’ abundances. Dispersion metrics
calculate the phylogenetic biodiversity expected given a source pool. Finally, dissimilarity differ from the former as they directly calculate the pairwise difference in 
phylogenetic biodiversity between assemblages. 
Table \ref{metricTable}
gives an overview of the functions contained within \emph{pez}, \emph{trait.asm}
permits the direct comparison of phylogenetic and trait structure,
following the \emph{traitgram} framework of XXX. Speeding and easing
the comparison of community phylogenetic metrics is important, since
each can reveal radically different aspects of the structure of
ecological assemblages \autocite{Cadotte2010}.
\begin{table}
\begin{center}
\begin{tabular}{p{3.5cm} p{14cm}}
  Function & Description\\\hline
  \emph{comparative.comm} & Stores community, phylogenetic, environmental, and species trait data\\
  \emph{shape} & Calcuates \emph{PSV}$^*$ \autocite{Helmus2007}, \emph{PSR}$^*$ \autocite{Helmus2007}, \emph{PD}$^*$ \& \emph{MPD}$^*$ \autocite{Faith1992}, Colless' Index$^*$ \autocite{Colless1982}, $\gamma$$^*$ \autocite{Pybus2000}, $\Delta$$^*$ \autocite{Warwick1995}, $E_{ED}$ \& $H_{ed}$ \autocite{Cadotte2010}, phylo-eigenvectors$^*$ \autocite{Diniz-Filho2011}\\
  \emph{evenness} & Calculates $\Delta$ \autocite{Warwick1995}, Phylogenetic Entropy \autocite{Allen2009}, \emph{PAE}, \emph{IAC}, $H_{aed}$, \& $E_{aed}$ \autocite{Cadotte2010}, Rao's quadratic entropy$^*$ \autocite{Rao1982a}, $\lambda$ \autocite{Pagel1999}, $\delta$ \autocite{Pagel1999}, $\kappa$ \autocite{Pagel1999}, Simpson's XXX (XXX)\\
  \emph{dispersion} & $SES_{MPD}$/\emph{NRI}$^*$ \autocite{Webb2000,Webb2002,Kembel2009}, $SES_{MNTD}/\emph{NTI}$$^*$ \autocite{Webb2000,Webb2002,Kembel2009}, \emph{INND} \autocite{Ness2011}, \emph{D}$^*$ \autocite{Fritz2010}\\
\emph{dissimilarity} &  UniFrac \autocite{Lozupone2005}, \emph{PCD} \autocite{Helmus2010}, PhyloSor \autocite{Bryant2008}, Rao's Q \autocite{Rao1982a}\\
  \emph{fingerprint.regression} & Compares phylogenetic, community co-existence, and trait similarity matrices using Mantel tests, quantile regressions, or linear models following \autocite{Cavender-Bares2004,Cavender-Bares2006}. Completely re-implements all features previously available in \emph{EcoPhyl} (XXX) \\
  \emph{scape} & Simulates community phylogenetic structure across a landscape, simulating phylogenetic repulsion, attraction, niche width, and range size (Helmus2012)\\ %STOPPED STOPPED
  \emph{trait.asm} & Calculate optimal traitgram that explains community data, following XXX\\
  Others? & classic.phy.signal (worth keeping?) - some stuff from willeerd?\\
  \hline
\end{tabular}
\caption{Some of the functions available in \emph{pez}. $^*$ indicates
  a metric whose code is based on existing \emph{R} code. Note that
  calculation of Pagel's $\lambda$ \autocite{Pagel1999}, $\delta$
  \autocite{Pagel1999}, $\kappa$ \autocite{Pagel1999} are calculated
  using the \emph{pgls} function in \emph{caper} \autocite{Orme2013}.}
\label{metricTable}
\end{center}
\end{table}
\subsubsection*{Models}
Model-based approaches are, in many ways, an advance over metrics that
simply quantify phylogenetic pattern in ecological assemblages. Models
explain phylogenetic structure across a number of assemblages
simultaneously, maximising statistical power, and can incorporate
phylogenetic, environmental, trait, and other information. \emph{Pez}
contains a complete re-implentation of all routines previously
available in \emph{EcoPhyl} (XXX) that regress species co-existence
matrices against environmental and, trait, and phylogenetic distance
matrices \autocite{Cavender-Bares2004,Cavender-Bares2006}. It extends
these routines to include, for example, more modern measures of
phylogenetic signal (\emph{e.g.}, $\lambda$, $delta$, $\kappa$, $D$),
and to examine differences in relationships among clades.
\subsection*{Simulation}
Finally, \emph{pez} contains a number of functions that will simulate
ecological communities under various null models of community assembly
and/or evolution. \emph{scape} is intended to model community assembly
over a region, and models both species niche width and
range. \emph{metapop.sim} fits models of trait evolution and dispersal
ability to a static landscape, evolving both a phylogeny, species
traits, and the resulting distribution of species across the
landscape. While these are works in progress, we are confident that
they at least form a basis for further model-based exploration of
data.
\subsection*{Acknowledgements}
MRH was funded by XXX; WDP was funded by XXX; both were funded by
SESYNC (XXX). We are grateful to A.\ Purvis and J.\ Cavender-Bares,
who contributed to metric framework according to which this package's
metrics are organised.

\printbibliography

\end{document}
