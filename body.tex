\documentclass{bioinfo}
\copyrightyear{2014}
\pubyear{2014}
\bibliographystyle{besjournals}
\begin{document}
\firstpage{1}

\title[\emph{pez}]{\emph{pez}: Phylogenetics for the Environmental
  Sciences} \author[Sample \textit{et~al}]{William D.\
  Pearse\,$^{1}$\footnote{to whom correspondence should be addressed},
  Marc Cadotte\,$^{2}$, Caroline Tucker\,$^{3}$, Steve Walker\,$^{4}$
  and Matthew R.\ Helmus\,$^5$} \address{$^{1}$Department of Ecology,
  Evolution, and Behavior, University of Minnesota, 100 Ecology
  Building, 1987 Upper Buford Circle, Saint Paul, Minnesota, 55108,
  USA, $^{2}$Department of Biological Sciences, University of
  Toronto–Scarborough, 1265 Military Trail, Scarborough, Ontario M1C
  1A4, Canada $^{3}$Department of Ecology and Evolutionary Biology,
  University of Colorado, Boulder, CO, USA, $^{4}$Ecology and
  Evolutionary Biology, University of Toronto, Toronto, Ontario M5S
  3G5, Canada, $^{5}$Department of Animal Ecology, Vrije Universiteit,
  1081 HV, Amsterdam, The Netherlands} \history{} \editor{}
\maketitle
\begin{abstract}
\section{Summary:}
\emph{pez} is an \emph{R} package that calculates $>$30 community
phylogenetic metrics, statistically models phylogenetic, trait, and
community data, and simulates community structure. It provides a
common programmatic standard for the management and manipulation of
these different data streams in \emph{R}.
\section{Availability:}
\emph{pez} is released under the GPL v3 open-source license, available
on the Internet from CRAN
(\href{http://cran.r-project.org}{http://cran.r-project.org}). The
package is under active development, and the authors welcome
contributions (see
\href{http://github.com/willpearse/pez}{http://github.com/willpearse/pez}).
\section{Contact:} \href{wdpearse@umn.edu}{William D.\ Pearse; wdpearse@umn.edu}
\end{abstract}
\section{Introduction}
Community phylogenetics (or eco-phylogenetics) is a sub-field of
ecology which links ecological phenomena with the evolutionary
processes that generate species and their traits
\citep[see][]{Webb2002,Cavender-Bares2009}. This growing field has
generated a number of statistical tools and software code and packages
that implement them
\citep[\emph{e.g.},][]{Webb2008,Regetz2009,Kembel2010,Orme2013,Eastman2013}. This
code is disparate and handles data differently, making routine data
analyses challenging. In many cases, the statistics are also not
formally implemented in a particular software package and are
available only as supplementary materials to papers. Without active
(public) maintenance, these valuable techniques are effectively lost
to the scientific community.

\emph{pez} provides an \emph{R} \citep{R2014} class
(\emph{comparative.comm}) that integrates phylogenetic, community,
environmental, and trait data in a single object. Table
\ref{metricTable} gives an overview of its features. Using a
\emph{comparative.comm} object, one can calculate a large body of
phylogenetic biodiversity metrics, many of which were previously
unavailable in \emph{R}. These metrics are grouped according to the
framework outlined in \citet{Pearse2014review}, easing the
interpretation and comparison of different aspects of phylogenetic
biodiversity. Further, \emph{pez} implements and extends the
regression framework presented by \citep{Cavender-Bares2004}, and
provides functions to simulate phylogenetic and ecological data. It is
our hope that \emph{pez} will provide an easy to use, common, and
extendable \emph{R} framework for eco-phylogenetic analysis.
\section{Description}
\subsection{Data manipulation and storage}
\emph{pez} extends the data manipulation and import/export features of
\emph{R}, and contains a unified class to contain all eco-phyogenetic
data. It simplifies the process of ensuring community, trait,
environmental, and phylogenetic data are compatible, and makes it easy
to remove species/sites from all datasets simultaneously. \emph{pez}
inherits from \emph{caper}'s \citep{Orme2013} \emph{comparative.data}
class, aiding comparative analysis of species trait data.

\subsection{Metrics}
Following the classification of \citet{Pearse2014review}, \emph{pez}
simplifies the calculation and comparison of a number of metrics by
grouping them into four categories: \emph{shape}, \emph{evenness},
\emph{dispersion}, and \emph{dissimilarity}. Shape metrics measure the
structure of an community phylogeny, while evenness metrics
additionally incorporate species abundances. Dispersion metrics
calculate examine whether phylogenetic biodiversity in an assemblage
differs from the expectation of random assembly from a given set of
species. Finally, dissimilarity measures the pairwise difference in
phylogenetic biodiversity between assemblages. The \emph{traitgram}
framework of \citet{Cadotte2013}, which directly compares phylogenetic
and functional trait community structure, is also included. Speeding
and easing the comparison of community phylogenetic metrics is
important, since different metrics can reveal separate aspects
eco-phylogenetic structure \citep{Cadotte2010}.

\begin{table*}
  \processtable{Overview of some of the functions available in \emph{pez}.\label{metricTable}}
  {\begin{tabular}{p{2.5cm} p{15cm}}\toprule
      Function & Description\\\midrule
      \emph{comparative.comm} & Stores community, phylogenetic, environmental, and species trait data\\
      \emph{shape} & Calculates \emph{PSV} \citep{Helmus2007}, \emph{PSR} \citep{Helmus2007}, \emph{PD} \& \emph{MPD} \citep{Faith1992}, Colless' Index \citep{Colless1982}, $\gamma$ \citep{Pybus2000}, $\Delta$ \citep{Warwick1995}, $E_{ED}$ \& $H_{ed}$ \citep{Cadotte2010}, phylo-eigenvectors \citep{Diniz-Filho2011}\\
      \emph{evenness} & Calculates $\Delta$ \citep{Warwick1995}, Phylogenetic Entropy \citep{Allen2009}, \emph{PAE}, \emph{IAC}, $H_{aed}$, \& $E_{aed}$ \citep{Cadotte2010}, Rao's quadratic entropy \citep{Rao1982a}, $\lambda$ \citep{Pagel1999}, $\delta$ \citep{Pagel1999}, $\kappa$ \citep{Pagel1999}, Simpson's XXX (XXX)\\
      \emph{dispersion} & Calculates $SES_{MPD}$/\emph{NRI} \citep{Webb2000,Webb2002,Kembel2009}, $SES_{MNTD}/\emph{NTI}$ \citep{Webb2000,Webb2002,Kembel2009}, \emph{INND} \citep{Ness2011}, \emph{D} \citep{Fritz2010}\\
      \emph{dissimilarity} &  Calculates UniFrac \citep{Lozupone2005}, \emph{PCD} \citep{Helmus2010}, PhyloSor \citep{Bryant2008}, Rao's Q \citep{Rao1982a}\\
      \emph{fingerprint.regression} & Compares phylogenetic, community co-existence, and trait similarity matrices using Mantel tests, quantile regressions, or linear models following \citep{Cavender-Bares2004,Cavender-Bares2006} \\
      \emph{scape} & Simulates community phylogenetic structure across a landscape, simulating phylogenetic repulsion, attraction, niche width, and range size \citep{Helmus2012}\\
      \emph{trait.asm} & Calculates optimal \emph{traitgram} that explains community data, following \citep{Cadotte2013}\\
      \botrule
\end{tabular}}{}
\end{table*}
\subsection{Models and Simulations}
\emph{pez} implements the \citet{Cavender-Bares2004} regression
framework for comparing species co-existence, environmental, trait,
and phylogenetic distance matrices, and extends it by including more
distance metrics and measures of phylogenetic signal. Developing new
techniques requires the simulation and examination of null
communities. \emph{pez} facilities this by containing functions to
simulate phylogenetically structured community assembly over a region,
and the evolution of species and their traits under iterated community
assembly.
\section*{Acknowledgement}
A.\ Purvis and J.\ Cavender-Bares contributed to metric framework
according to which this package's metrics are organised.
\paragraph{Funding\textcolon}MRH was funded by XXX; WDP was funded by
XXX; both were funded by SESYNC (XXX).

\begin{thebibliography}{30}
\providecommand{\natexlab}[1]{#1}
\providecommand{\url}[1]{\texttt{#1}}
\providecommand{\urlprefix}{URL }

\bibitem[{Allen \emph{et~al.}(2009)Allen, Kon \& Bar-Yam}]{Allen2009}
Allen, B., Kon, M. \& Bar-Yam, Y. (2009) A new phylogenetic diversity measure
  generalizing the shannon index and its application to phyllostomid bats.
  \emph{The American Naturalist} \textbf{174}, 236--243.

\bibitem[{Bryant \emph{et~al.}(2008)Bryant, Lamanna, Morlon, Kerkhoff, Enquist
  \& Green}]{Bryant2008}
Bryant, J.A., Lamanna, C., Morlon, H., Kerkhoff, A.J., Enquist, B.J. \& Green,
  J.L. (2008) Microbes on mountainsides: contrasting elevational patterns of
  bacterial and plant diversity. \emph{Proceedings of the National Academy of
  Sciences} \textbf{105}, 11505--11511.

\bibitem[{Cadotte \emph{et~al.}(2013)Cadotte, Albert \& Walker}]{Cadotte2013}
Cadotte, M., Albert, C.H. \& Walker, S.C. (2013) The ecology of differences:
  assessing community assembly with trait and evolutionary distances.
  \emph{Ecology Letters} \textbf{16}, 1234--1244.

\bibitem[{Cadotte \emph{et~al.}(2010)Cadotte, Davies, Regetz, Kembel, Cleland
  \& Oakley}]{Cadotte2010}
Cadotte, M.W., Davies, T.J., Regetz, J., Kembel, S.W., Cleland, E. \& Oakley,
  T.H. (2010) Phylogenetic diversity metrics for ecological communities:
  integrating species richness, abundance and evolutionary history.
  \emph{Ecology Letters} \textbf{13}, 96--105.

\bibitem[{Cavender-Bares \emph{et~al.}(2006)Cavender-Bares, Keen \&
  Miles}]{Cavender-Bares2006}
Cavender-Bares, J., Keen, A. \& Miles, B. (2006) {Phylogenetic structure of
  Floridian plant communities depends on taxonomic and spatial scale}.
  \emph{Ecology} \textbf{87}, S109--S122.

\bibitem[{Cavender-Bares \emph{et~al.}(2004)Cavender-Bares, Ackerly, Baum \&
  Bazzaz}]{Cavender-Bares2004}
Cavender-Bares, J., Ackerly, D.D., Baum, D.a. \& Bazzaz, F.a. (2004)
  {Phylogenetic overdispersion in Floridian oak communities}. \emph{The
  American Naturalist} \textbf{163}, 823--43.

\bibitem[{Cavender-Bares \emph{et~al.}(2009)Cavender-Bares, Kozak, Fine \&
  Kembel}]{Cavender-Bares2009}
Cavender-Bares, J., Kozak, K., Fine, P.V.A. \& Kembel, S.W. (2009) {The merging
  of community ecology and phylogenetic biology}. \emph{Ecology letters}
  \textbf{12}, 693--715.

\bibitem[{Colless(1982)}]{Colless1982}
Colless, D.H. (1982) Review of phylogenetics: the theory and practice of
  phylogenetic systematics. \emph{Systematic Zoology} \textbf{31}, 100--104.

\bibitem[{Diniz-Filho \emph{et~al.}(2011)Diniz-Filho, Cianciaruso, Rangel \&
  Bini}]{Diniz-Filho2011}
Diniz-Filho, J.A.F., Cianciaruso, M.V., Rangel, T.F. \& Bini, L.M. (2011)
  Eigenvector estimation of phylogenetic and functional diversity.
  \emph{Functional Ecology} \textbf{25}, 735--744.

\bibitem[{Eastman \emph{et~al.}(2013)Eastman, Paine \& Hardy}]{Eastman2013}
Eastman, J., Paine, T. \& Hardy, O. (2013) \emph{spacodiR: Spatial and
  Phylogenetic Analysis of Community Diversity}.
  \urlprefix\url{http://CRAN.R-project.org/package=spacodiR}.

\bibitem[{Faith(1992)}]{Faith1992}
Faith, D.P. (1992) Conservation evaluation and phylogenetic diversity.
  \emph{Biological Conservation} \textbf{61}, 1--10.

\bibitem[{Fritz \& Purvis(2010)}]{Fritz2010}
Fritz, S.A. \& Purvis, A. (2010) selectivity in mammalian extinction risk and
  threat types: a new measure of phylogenetic signal strength in binary traits.
  \emph{Conservation Biology} \textbf{24}, 1042--1051.

\bibitem[{Helmus \& Ives(2012)}]{Helmus2012}
Helmus, M.R. \& Ives, A.R. (2012) Phylogenetic diversity-area curves.
  \emph{Ecology} \textbf{93}, S31--S43.

\bibitem[{Helmus \emph{et~al.}(2010)Helmus, Keller, Paterson, Yan, Cannon \&
  Rusak}]{Helmus2010}
Helmus, M.R., Keller, W., Paterson, M.J., Yan, N.D., Cannon, C.H. \& Rusak,
  J.A. (2010) Communities contain closely related species during ecosystem
  disturbance. \emph{Ecology Letters} \textbf{13}, 162--174.

\bibitem[{Helmus \emph{et~al.}(2007)Helmus, Savage, Diebel, Maxted \&
  Ives}]{Helmus2007}
Helmus, M.R., Savage, K., Diebel, M.W., Maxted, J.T. \& Ives, A.R. (2007)
  Separating the determinants of phylogenetic community structure.
  \emph{Ecology Letters} \textbf{10}, 917--925.

\bibitem[{Kembel(2009)}]{Kembel2009}
Kembel, S.W. (2009) {Disentangling niche and neutral influences on community
  assembly: assessing the performance of community phylogenetic structure
  tests}. \emph{Ecology Letters} \textbf{12}, 949--60.

\bibitem[{Kembel \emph{et~al.}(2010)Kembel, Cowan, Helmus, Cornwell, Morlon,
  Ackerly, Blomberg \& Webb}]{Kembel2010}
Kembel, S.W., Cowan, P.D., Helmus, M.R., Cornwell, W.K., Morlon, H., Ackerly,
  D.D., Blomberg, S.P. \& Webb, C.O. (2010) Picante: R tools for integrating
  phylogenies and ecology. \emph{Bioinformatics} \textbf{26}, 1463--1464.

\bibitem[{Lozupone \& Knight(2005)}]{Lozupone2005}
Lozupone, C. \& Knight, R. (2005) {UniFrac}: a new phylogenetic method for
  comparing microbial communities. \emph{Applied and Environmental
  Microbiology} \textbf{71}, 8228--8235.

\bibitem[{Ness \emph{et~al.}(2011)Ness, Rollinson \& Whitney}]{Ness2011}
Ness, J.H., Rollinson, E.J. \& Whitney, K.D. (2011) Phylogenetic distance can
  predict susceptibility to attack by natural enemies. \emph{Oikos}
  \textbf{120}, 1327--1334.

\bibitem[{Orme \emph{et~al.}(2013)Orme, Freckleton, Thomas, Petzoldt, Fritz,
  Isaac \& Pearse}]{Orme2013}
Orme, D., Freckleton, R., Thomas, G., Petzoldt, T., Fritz, S., Isaac, N. \&
  Pearse, W.D. (2013) \emph{caper: comparative analyses of phylogenetics and
  evolution in {R}}. \urlprefix\url{http://CRAN.R-project.org/package=caper}, r
  package version 0.5.2.

\bibitem[{Pagel(1999)}]{Pagel1999}
Pagel, M. (1999) {Inferring the historical patterns of biological evolution}.
  \emph{Nature} \textbf{401}, 877--884.

\bibitem[{Pearse \emph{et~al.}(2014)Pearse, Cavender-Bares, Puvis \&
  Helmus}]{Pearse2014review}
Pearse, W.D., Cavender-Bares, J., Puvis, A. \& Helmus, M.R. (2014) Metrics and
  models of community phylogenetics. \emph{Modern Phylogenetic Comparative
  Methods and their Application in Evolutionary Biology---Concepts and
  Practice} (ed. L.Z. Garamszegi), Springer-Verlag, Berlin, Heidelberg.

\bibitem[{Pybus \& Harvey(2000)}]{Pybus2000}
Pybus, O. \& Harvey, P. (2000) Testing macro-evolutionary models using
  incomplete molecular phylogenies. \emph{Proceedings of the Royal Society B:
  Biological Sciences} \textbf{267}, 2267--2272.

\bibitem[{{R Core Team}(2014)}]{R2014}
{R Core Team} (2014) \emph{R: A language and environment for statistical
  computing}. R Foundation for Statistical Computing, Vienna, Austria.

\bibitem[{Rao(1982)}]{Rao1982a}
Rao, C. (1982) Diversity and dissimilarity coefficients: a unified approach.
  \emph{Theoretical Population Biology} \textbf{21}, 24--43.

\bibitem[{Regetz \emph{et~al.}(2009)Regetz, Cadotte \& Davies}]{Regetz2009}
Regetz, J., Cadotte, M. \& Davies, J. (2009) \emph{ecoPD: Ecologically-informed
  phylodiversity metrics}.
  \urlprefix\url{\url{http://R-Forge.R-project.org/projects/ecopd/}}.

\bibitem[{Warwick \& Clarke(1995)}]{Warwick1995}
Warwick, R.M. \& Clarke, K.R. (1995) New `biodiversity' measures reveal a
  decrease in taxonomic distinctness with increasing stress. \emph{Marine
  Ecology Progress Series} \textbf{129}.

\bibitem[{Webb(2000)}]{Webb2000}
Webb, C.O. (2000) {Exploring the phylogenetic structure of ecological
  communities: an example for rain forest trees}. \emph{The American
  Naturalist} \textbf{156}, 145--155.

\bibitem[{Webb \emph{et~al.}(2008)Webb, Ackerly \& Kembel}]{Webb2008}
Webb, C.O., Ackerly, D.D. \& Kembel, S.W. (2008) {Phylocom: software for the
  analysis of phylogenetic community structure and trait evolution.}
  \emph{Bioinformatics} \textbf{24}, 2098--100.

\bibitem[{Webb \emph{et~al.}(2002)Webb, Ackerly, McPeek \& Donoghue}]{Webb2002}
Webb, C.O., Ackerly, D.D., McPeek, M.A. \& Donoghue, M.J. (2002) Phylogenies
  and community ecology. \emph{Annual Review of Ecology and Systematics}
  \textbf{33}, 475--505.

\end{thebibliography}
\end{document}